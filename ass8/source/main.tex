\documentclass{article}
\usepackage[utf8]{inputenc}
\usepackage{karnaugh-map}

% ADD TITLE HERE
\title{K-map for (a)}
\author{K ganesh reddy}
\date{7/01/2021}

\begin{document}

\maketitle

\section{expression for a=D'B'(C'A+CA')}

\begin{karnaugh-map}[4][4][1][][]
    \maxterms{0,1,3,4,5,6,7,9,10,11,12,13,14,15}
    \minterms{2,8}
   
    
    
    \implicant{1}{11}
    \implicant{4}{14}
    \implicant{0}{5}
     \implicant{15}{10}
    % note: position for start of \draw is (0, Y) where Y is
    % the Y size(number of cells high) in this case Y=2
    \draw[color=black, ultra thin] (0, 4) --
    node [pos=0.7, above right, anchor=south west] {$CD$} % YOU CAN CHANGE NAME OF VAR HERE, THE $X$ IS USED FOR ITALICS
    node [pos=0.7, below left, anchor=north east] {$AB$} % SAME FOR THIS
    ++(135:1);
        
    \end{karnaugh-map}
    
    
    From the K-Map above, we can infer that the POS expression is:
\begin{equation}
    c=D'B'(C'A+CA')
\end{equation}

\end{document}
